\chapter{Sonstiges}

\section{Vorwort}

\lipsum[1-5]
\newpage





\section{Trinkbares}


\subsection{Honig-Zitronen-Ingwer Shot}

\textbf{Infekte} und \textbf{Erkältungskrankheiten} und zur \textbf{Stärkung des Immunsystems}

Für ca 1,5-2 Liter:
\begin{itemize}
	\item 3 Äpfel
	\item 8 Orangen
	\item 2 Zitronen
	\item 200g Ingwer
	\item 2 Löffel Kurkuma
	\item kräftige Prise schwarzer Pfeffer
	\item Honig
\end{itemize}

Klein schneiden, auspressen und in den Mixer geben. Mit Honig süßen bis die Schärfe des Ingwers individuell angenehm ist.

\url{http://hp.rudolphrichard.de/?p=601}

\subsection{Met}

\subsection{Naturtrüber Absinth Erdgeist}

'Du, Geist der Erde, bist mir näher; schon fühl ich meine Kräfte höher'

\begin{itemize}
	\item 0,8l 85-98\% Trinkalkohol oder 40\%igen Korn
	\item 30g Wermut (Artemisia absinthium)
	\item 8,5g Ysop (Hyssopus officinalis)
	\item 1,8g Kalmuswurzel (Acorus calamus)
	\item 6g Zitronenmelisse (Melissa officinalis)
	\item 30g Anis (Pimpinella anisum)
	\item 25g Fenchelsamen (Foeniculum vulgare)
	\item 10g Sternanis (Illiciu verum)
	\item 3,2g Koriandersamen (Coriandrum sativum)
\end{itemize}

Kräuter grob mahlen und in ein verschließbares Gefäß geben.
Alkohol hinzugeben.
Schütteln.
7 Tage im Dunkeln ziehen lassen und immer wieder schütteln.
Abschließend erneut schütteln und dann Kräuter abfiltern.
Kräuter ausdrücken um den Rest Alkohol herauszuholen.
Absinth mit 0,25 Liter Wasser verdünnen.
Lichtgeschützt unter 30°C aufbewahren.


Ausdrücklich kein Heil- sondern ein Rauschmittel



\section{Pasten}

\subsection{Paste für die Schilddrüse}
 
Nachfolgendes ist noch nicht ausgetestet, soll aber bei einer Schilddrüseninsuffizienz unterstützen und bis zu die Einnahme von Schilddrüsenhormonen substituieren.

\begin{itemize}
	\item 2TL Kurkuma
	\item 1TL Ashwaganda
	\item 1TL Hagebuttenpulver
	\item 1 Tropfen gutes Öl
	\item Schwarzer Pfeffer
\end{itemize}

Als Paste anrühren, mit Wasser oder Saft auffüllen



\section{Pflanzenteile}


\subsection{Aloe Vera}

\textbf{Sonnenbrand:}


\subsection{Goethepflanze}


\subsection{Kapuzinerkresse}

\textbf{Infekte} und \textbf{Erkältungskrankheiten} und zur \textbf{Stärkung des Immunsystems}


\subsection{Spitzwegerich}

Der Spitzwegerich wird auch das Feld- und Wiesenpflaster genannt. Der Wirkstoff Aucubin wirkt desinfizierend, antibiotisch und wundverschließend.\\
Bei \textbf{Wunden} und \textbf{Insektenstichen} mehrere Blätter zu einem Knoten verknoten, in der Hand reiben und drücken. Den Knoten dann auf der Wunde ausdrücken. ~\cite{swrhandwerkskunst}  \url{https://www.youtube.com/watch?v=OQeWS5Ktl0Y}

