\chapter{Salben}

\section{Vorwort}

% TODO

\lipsum[1-2]




\section{Herstellung allgemein}

Eine Salbe kann entweder aus einem langsam ausgezogenen Öl, oder aus einem Warmauszug gewonnen werden. Das Öl wird in jedem Fall vorsichtig erhitzt auf maximal 40°C. Danach wird eine kleine Menge Bienenwachs und optional schluckweise Tinkturen zugegeben. Um zu testen, ob genügend Wachs zugegeben wurde, gibt man einfach eine kleine Menge auf einen kühlen Teller und lässt das ganze etwas sitzen. Nach einer kurzen Aushärtungszeit kann man die Konsistenz der Salbe testen. Je mehr Wachs zugegeben wird, desto härter wird die Salbe und lässt sich weniger gut verstreichen.


\newpage

\section{Johanniskrautsalbe}

% TODO

\index{Winterdepression} \index{Lichtmangel} \index{Depressive Stimmung} \index{Schlaflosigkeit} \index{Angstzustände} \index{Blutergüsse}
\index{Entzündungen} \index{Prellungen} \index{Wunden} \index{Juckreiz} \index{Sonnenbrand} \index{Schmerzen} \index{Schwellungen} \index{Wundheilung} \index{Wundversorgung} \index{Massage} \index{Muskelentspannung} \index{Winterdepression} \index{Antidepressivum} \index{Narbenentfernung} \index{Narben} \index{Verspannungen}

\subsection{Anwendung}
% TODO
\subsection{Herstellung}
% TODO

%\subsection{Anmerkungen}



\newpage




\section{Gierschsalbe}

% TODO

\url{https://www.mitliebegemacht.at/gierschsalbe-altes-hausmittel-bei-rheuma-gicht-arthrose/}

\index{Rheuma} \index{Gicht} \index{Arthrose} \index{Blasenentzündung} \index{Ischiasbeschwerden}

\subsection{Anwendung}
% TODO
\subsection{Herstellung}
% TODO
Gierschöl und Bienenwachs und langsam schmelzen. Nach dem schmelzen vom Herd nehmen, warm rühren und schluckweise Gierschtinktur zugeben.

%\subsection{Anmerkungen}



\newpage




\section{Holundersalbe}

% TODO

\index{Insektenstiche} \index{Hautirritation} \index{Juckreiz} \index{Mückenstiche}

\subsection{Anwendung}
% TODO
\textbf{Mückenstiche:} \\ 

\textbf{Hautirritationen:} \\ 

\textbf{Juckreiz:} \\ 

\subsection{Herstellung}
% TODO

%\subsection{Anmerkungen}





\newpage




\section{Wundsalbe}

Eine einfache und altbewährt Salbe bei oberflächlichen, stumpfen und geschlossenen Wunden wie zB Prellungen und blauen Flecken. Die Salbe ist unter anderem im Militär ein Geheimtipp, zB bei Prellungen vom Kugeltreffern unter kugelsichereren Vesten.

% TODO Quellen

\index{Wunden} \index{Prellungen}

\subsection{Anwendung}

Bei Wunden, Prellungen und anderen stumpfen verschlossenen Wunden kann diese Salbe angewandt werden, um diese zu lindern. Dazu wird sie auf die betreffende Stelle aufgetragen und sanft einmassiert.

\subsection{Herstellung}

Die Salbe besteht aus nur drei Zutaten, Butter, Honig und Salz. Diese werden miteinander vermengt und zu einer Salbe verrührt.

\begin{itemize}
	\item Butter
	\item Honig
	\item Salz
\end{itemize}

%\subsection{Anmerkungen}



