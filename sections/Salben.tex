\chapter{Salben}

\section{Vorwort}

% TODO

\lipsum[1-2]




\section{Herstellung allgemein}

Eine Salbe kann entweder aus einem langsam ausgezogenen Öl, oder aus einem Warmauszug gewonnen werden. Das Öl wird in jedem Fall vorsichtig erhitzt auf maximal 40°C. Danach wird eine kleine Menge Bienenwachs und optional schluckweise Tinkturen zugegeben. Um zu testen, ob genügend Wachs zugegeben wurde, gibt man einfach eine kleine Menge auf einen kühlen Teller und lässt das ganze etwas sitzen. Nach einer kurzen Aushärtungszeit kann man die Konsistenz der Salbe testen. Je mehr Wachs zugegeben wird, desto härter wird die Salbe und lässt sich weniger gut verstreichen.



\section{Johanniskrautsalbe}

% TODO

\subsection{Anwendung}
% TODO
\subsection{Herstellung}
% TODO
\subsection{Anmerkungen}


\newpage


\section{Gierschsalbe}

% TODO

\url{https://www.mitliebegemacht.at/gierschsalbe-altes-hausmittel-bei-rheuma-gicht-arthrose/}

\index{Rheuma} \index{Gicht} \index{Arthrose} \index{Blasenentzündung} \index{Ischiasbeschwerden}

\subsection{Anwendung}
% TODO
\subsection{Herstellung}
% TODO
Gierschöl und Bienenwachs und langsam schmelzen. Nach dem schmelzen vom Herd nehmen, warm rühren und schluckweise Gierschtinktur zugeben.

\subsection{Anmerkungen}


\newpage


\section{Holundersalbe}

% TODO

\index{Insektenstiche} \index{Hautirritation} \index{Juckreiz} \index{Mückenstiche}

\subsection{Anwendung}
% TODO
\textbf{Mückenstiche:} \\ 

\textbf{Hautirritationen:} \\ 

\textbf{Juckreiz:} \\ 

\subsection{Herstellung}
% TODO
\subsection{Anmerkungen}





\newpage




\section{Wundsalbe}

% TODO

\index{Wunden} \index{Prellungen}

\subsection{Anwendung}
% TODO
\subsection{Herstellung}
% TODO
Butter Honig Salz

\subsection{Anmerkungen}



