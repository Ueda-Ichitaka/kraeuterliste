
%% ==============================
\chapter{Präambel}
%% ==============================
\label{   }
%% ==============================


%% ==============================
\section{Arbeitsweise}
%% ==============================
\label{   }
%% ==============================

Generell ist eine labortechnische Arbeitsweise zu empfehlen, das bedeutet alle Oberflächen sind sauber (bestenfalls steril, einfach mit Wasser und Spüli oder Isopropanol abgewischt) alle verwendeten Geräte sind sauber (hier genügt Spülmaschinensauber) und man verwendet Laborhandschuhe. Eine Schutzbrille ist dann von Nöten, wenn es durch den Vorgang zu Spritzern kommen kann, die zB die Augen treffen können. Haare sind zurückzubinden, Bärte vorher auszukemmen.\\
Generell sollte jede Lösung beim abgießen filtriert werden, hierbei empfielt sich das mehrfache filtrieren. Geeignet sind zB. sehr feine Teefilter. Pflanzenteile in den Lösungen werden vor der Entsorgung ausgedrückt oder gepresst, um weitere Lösung zu erhalten, die aufgefangen und filtriert werden muss.\\
Pflanzen sollten generell nur mit Keramikmessern ab- und zerschnitten werden, um eine Reaktion der Wirkstoffe mit den Metallionen aus herkömmlichen Messern zu vermeiden.\\



%% ==============================
\section{Lagerung}
%% ==============================
\label{   }
%% ==============================

Die Präparate sollten möglichst dunkel, kühl, trocken und in sterilen Gefäßen gelagert werden, um ihre Haltbarkeit zu garantieren. Dazu empfehlen sich dunkle Flaschen.



%% ==============================
\section{Haltbarkeit}
%% ==============================
\label{   }
%% ==============================

Die Haltbarkeit ist je nach Präparat und Lösungsträger verschieden. Alkoholische Tinkturen halten sich problemlos mehrere Jahre. Öle halten sich abhängig vom verwendeten Öl, wobei Olivenöl mit 2 Jahren die längste Haltbarkeit aufweist. Salben halten bei dunkler, trockener Lagerung mindestens 1 Jahr. Ist das Präparat ranzig geworden, ist es unbrauchbar.




%% ==============================
\section{Grundlagen der Wundversorgung}
%% ==============================
\label{   }
%% ==============================

% TODO



%% ==============================
\subsection{Oberflächliche Wunden}
%% ==============================
\label{   }
%% ==============================

% TODO


%% ==============================
\subsection{Tiefere Wunden}
%% ==============================
\label{   }
%% ==============================

% TODO


%% ==============================
\subsection{Innere Wunden}
%% ==============================
\label{   }
%% ==============================

% TODO



%% ==============================
\section{Auszüge}
%% ==============================
\label{   }
%% ==============================

% TODO umsortieren und umordnen

Als Tinkturen bezeichnet man Flüssigkeiten, in denen Wirkstoffe gelöst wurden. Diese Lösung kann auf verschiedene Weisen erfolgen, die häufigsten im Bereich der Heilpflanzen sind die folgenden:

\begin{itemize}
	\item Tee (Kräuteraufguss/Kräuterabsud)
	\item Ölauszug
	\item Ätherische Öle
	\item Alkoholische Extrakte
\end{itemize}

Der Vorgang der Lösung der Wirkstoffe wird auch als Extraktion bezeichnet, das entstehende Produkt als Extrakt. Die Vorgehensweisen bei der Herstellung solcher Lösungen sind größtenteils allgemeingültig und müssen nur bei manchen Pflanzen spezifisch angepasst werden. Hier soll zunächst im Detail auf die allgemeine Herstellung eingegangen werden.



%% ==============================
\section{Tee}
%% ==============================
\label{   }
%% ==============================

Die Bezeichnung \textit{Tee} steht korrekterweise nur für den Aufguss, der aus den Bestandteilen der Teepflanze zubereitet wird. Allerdings hat er sich im allgemeinen Sprachgebrauch für sämtliche Aufgüsse von Pflanzen etabliert. So soll auch in diesem Kontext keine Unterscheidung zwischen Tee und Kräuteraufguss gemacht werden, doch sei erwähnt, dass damit stets Kräuteraufgüsse gemeint sind.



%% ==============================
\subsection{Aufguss}
%% ==============================
\label{   }
%% ==============================

Ein Aufguss bezeichnet das Übergießen fester Drogen (beispielsweise Pflanzenteile wie Blätter oder Blüten) mit heißem bis kochendem Wasser. Im Kontext einer medizinischen Anwendung der Kräuter wird hierbei im Gegensatz zur Anwendung als Genussmittel das Kraut nicht bereits nach kurzer Zeit wieder aus dem Aufguss entfernt, sondern verweilt über einen längeren Zeitraum darin - das so genannte \textit{Ziehen}. Die exakte Dauer hängt hierbei ab vom Zerkleinerungsgrad der Droge, der Löslichkeit der gewünschten Wirkstoffe, sowie dem Verhältnis zwischen Drogen- und Wassermenge. Eine Maximaldauer gilt im Allgemeinen nicht, da Wirkstoffe bis zum Erreichen eines osmotischen Gleichgewichts oder der maximalen Löslichkeit der Stoffe aus den Drogen herausgetrennt werden. Spezifische Pflanzen können allerdings je nach Ziehzeit in der Wirkung variieren, da sich nicht alle Wirkstoff gleich schnell lösen.



%% ==============================
\subsubsection{Allgemeines Vorgehen}
%% ==============================
\label{   }
%% ==============================

Zum Herstellen eines medizinisch wirksamen Aufgusses werden bis zu ca. 5 Gramm der zerkleinerten Droge, entweder im frischen oder getrockneten Zustand, mit 200-300ml heißem oder kochendem Wasser übergossen. Soll eine ganze Kanne (>1l) angesetzt werden, kann bei getrockneten Pflanzen ein Esslöffel als Maßeinheit hergenommen werden. In einem solchen Fall sind ein bis zwei Esslöffel auf einen Liter Wasser im Allgemeinen ausreichend.



%% ==============================
\subsubsection{Hinweise}
%% ==============================
\label{   }
%% ==============================

Auch wenn viele Wirkstoffe in Wasser eine vergleichsweise geringe Löslichkeit besitzen, sollte dennoch auf das Vermeiden einer Überdosis geachtet werden. Hierzu ist die Gesamtmasse des verwendeten Rohmaterials (frisch oder getrocknet), sowie die spezifische Wirkungsweise und die Abbaugeschwindigkeit im Körper zu beachten.



%% ==============================
\subsection{Absud}
%% ==============================
\label{   }
%% ==============================

Ein Absud (auch Dekokt) ist ein in der Regel kalt angesetzer Auszug mit Pflanzenteilen, deren Wirkstoffe sich schwer durch den bloßen Aufguss mit heißem Wasser lösen. Im Gegensatz zum Aufguss ist der Absud oft ein mehrere Stunden dauernder Prozess, der sich besonders für Pflanzenteile wie umhüllte Früchte (z.B. getrocknete Datteln) oder Pflanzenteile von fester Struktur (z.B. Wurzeln und Stängel) eignet.



%% ==============================
\subsubsection{Allgemeines Vorgehen}
%% ==============================
\label{   }
%% ==============================

Die Pflanzenteile werden soweit möglich zerkleinert und in ein hitzebeständiges Gefäß mit kaltem oder lauwarmem Wasser gegeben. In diesem Gefäß wird die Mischung nun einige Zeit Ruhen gelassen (Wurzeln einige Stunden, eventuell eine Nacht; Früchte einige Minuten). Anschließend wird die Mischung unter Rühren bis zum Kochen erhitzt und soll anschließend weiterhin unter Rühren köcheln. Auch hier hängt die Dauer im Wesentlichen von den verwendeten Pflanzen und Pflanzenteilen ab. Je nach Präferenz können die Pflanzenteile nun noch gepresst werden bevor sie aus dem Absud gefiltert werden.
Die Anwendung eines Absuds erfolgt im Wesentlichen wie bei einem Aufguss.



%% ==============================
\subsubsection{Hinweise}
%% ==============================
\label{   }
%% ==============================

Ein Absud sollte nicht als verstärkte Variante eines Aufgusses betrachtet werden, da manche Wirkstoffe unter der langfristigen Einwirkung von Hitze zerfallen. Daher wird er angewandt wenn die gewünschten Wirkstoffe sich nicht durch das bloße Übergießen mit heißem Wasser lösen können. Entsprechend sollte man sich auch bei einem Aufguss informieren, wie lange die Wirkstoffe der Hitze ausgesetzt werden dürfen.

						


						
