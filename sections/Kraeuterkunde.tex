\chapter{Kräuterkunde}

\section{Vorwort}

In diesem Kapitel werden die Wirkung und die Wirkstoffe der einzelnen Kräuter und weiterer Pflanzenteile erklärt. Dies wird hier zentral gemacht, als zentrales Nachschlagewerk, damit nicht bei jedem Kräuterauszug die Wirkung erneut aufgezählt werden muss. Dennoch schließt das eine einzelne Auszählung nicht aus, besonders um den Auszug der Seiten einzeln eingenständig weitergeben zu können. Hierbei sollen die Aussagen mit Literaturquellen belegt werden. Es können wörtliche Übernehmungen vorkommen.

% TODO Wirkstoffe alphabetisch sortieren
% TODO Wirkungsweisen alphabetisch sortieren
% TODO Literaturnachweise alphabetisch sortieren
% TODO sections alphabetisch sortieren
% TODO Wermut, Senf, Wacholder, Hanf, Stechapfel, Steinklee, Rosskastanie


\section{Schafgarbe (Achillea millefolium)}

bla bla text

\textbf{Wirkstoffe:} Azulene, Gerbstoffe, Flavonoide, ätherische Öle\\

\textbf{Wirkungsweisen:} entzündungshemmend, wundheilend, desinfizierend\\

Literaturnachweise: \cite{nedoma2018heilsalben}

\section{Melisse}

bla bla

\section{Aloe Vera}

bla bla

\section{Goethepflanze}

bla bla

\section{Kapuzinerkresse}

bla bla

\section{Spitzwegerich}

bla bla

\section{Breitwegerich}

bla bla

\section{Frauenmantel}

bla bla

\section{Ringelblume (Calendula officinalis)}

bla bla

\textbf{Wirkstoffe:} Carotinoide, Flavonoide, Querecetin, Saponine, Salicylsäure\\

\textbf{Wirkungsweisen:} wundheilend, hautregenerierend, zusammenziehend, antibakteriell, entzündungshemmend, erweichend \\

Literaturnachweise: \cite{nedoma2018heilsalben}

\section{Minze}

bla bla

\section{Johanniskraut (Hypericum perforatum)}

Ein Kraut voller Kraft, man sagt, Johanniskraut speichert die Kraft der Sonne.

\textbf{Wirkstoffe:} Hypericin, Hyperforin, Flavonoide\\

\textbf{Wirkungsweisen:} zellschützend, antioxidativ, krebshemmend, antibakteriell, angstlösend, stimmunsaufhellend, lichtsensibilisierend, antidepressiv\\

Literaturnachweise: \cite{nedoma2018heilsalben}

\section{Thymian}

bla bla

\section{Salbei}

bla bla

\section{Kamille (Matricaria chamomilla)}

bla bla

\textbf{Wirkstoffe:} Chamazulen, Cumarine, Flavonoide, ätherische Öle\\

\textbf{Wirkungsweisen:} antibakteriell, wundheilend, beruhigend\\

Literaturnachweise: \cite{nedoma2018heilsalben}

\section{Cistrose}

bla bla

\section{Beifuß}

bla bla

\section{Ysop}

bla bla

\section{Giersch}

bla bla

\section{Bärlauch}

bla bla

\section{Lavendel}

bla bla

\section{Beiwell (Symphytum officinalis)}

bla bla

\textbf{Wirkstoffe:} Allantoin, Cholin, Schleimstoffe, Gerbstoffe, Flavonoide, Kieselsäure, Alkaloide\\

\textbf{Wirkungsweisen:} wundheilend, geweberegenerierend, abschwellend, schmerzlindernd, entzündungshemmend, erweichend, knochenwachstumsfördernd, blutstillend \\

Literaturnachweise: \cite{nedoma2018heilsalben}

\section{Malvenblüten (Malva)}

bla bla

\textbf{Wirkstoffe:} Schleimstoffe, Anthocyane, Flavonoide, Gerbstoffe, Saponine\\

\textbf{Wirkungsweisen:} feuchtigkeitsspendend, hautschützend, entzündungshemmend, juckreizstillend, zusammenziehend \\

Literaturnachweise: \cite{nedoma2018heilsalben}

\section{Rosenblüten (Rosa)}

bla bla

\textbf{Wirkstoffe:} Gerbstoffe, ätherische Öle\\

\textbf{Wirkungsweisen:} desinfizierend, zusammenziehend, entspannend, schmerzstillend, kühlend, abschwellend, entzündungshemmend, blutstillend \\

Literaturnachweise: \cite{nedoma2018heilsalben}

\section{Lindenblüten (Tilia)}

bla bla

\textbf{Wirkstoffe:} Schleimstoffe, Flavonoide, Gerbstoffe, Saponine, Farnesol, ätherische Öle \\

\textbf{Wirkungsweisen:} beruhigend, wärmend, abschwellend, entspannend, krampflösend, entzündungshemmend \\

Literaturnachweise: \cite{nedoma2018heilsalben}

\section{Gänseblümchen (Bellisperennis)}

bla bla

\textbf{Wirkstoffe:} Bitterstoffe, Saponin, Schleimstoffe\\

\textbf{Wirkungsweisen:} schmerzstillend, wundheilend, hautklärend\\

Literaturnachweise: \cite{nedoma2018heilsalben}

\section{Anis}

bla bla

\section{Kümmel}

bla bla

\section{Liebstöckel}

bla bla

\section{Holunder}

bla bla

\section{Himbeerblätter}

bla bla

\section{Erdbeerblätter}

bla bla

\section{Johannisbeerblätter}

bla bla

\section{Eichenrinde}

bla bla

\section{Schwarztee}

bla bla

\section{Angelikawurzel}

bla bla

\section{Perlagonienwurzel}

bla bla

\section{Ingwer}

bla bla

\section{Fenchelsamen}

bla bla

\section{Chili}

bla bla

\section{Honig}

bla bla

\section{Propolis}

bla bla

\section{Olivenöl}

bla bla


