\chapter{Kräuterkunde}

\section{Vorwort}

In diesem Kapitel werden die Wirkung und die Wirkstoffe der einzelnen Kräuter und weiterer Pflanzenteile erklärt. Dies wird hier zentral gemacht, als zentrales Nachschlagewerk, damit nicht bei jedem Kräuterauszug die Wirkung erneut aufgezählt werden muss. Dennoch schließt das eine einzelne Auszählung nicht aus, besonders um den Auszug der Seiten einzeln eingenständig weitergeben zu können. Hierbei sollen die Aussagen mit Literaturquellen belegt werden. Es können wörtliche Übernehmungen vorkommen.



\section{Schafgarbe}

bla bla

\section{Melisse}

bla bla

\section{Aloe Vera}

bla bla

\section{Goethepflanze}

bla bla

\section{Kapuzinerkresse}

bla bla

\section{Spitzwegerich}

bla bla

\section{Breitwegerich}

bla bla

\section{Frauenmantel}

bla bla

\section{Ringelblume}

bla bla

\section{Minze}

bla bla

\section{Johanniskraut}

bla bla

\section{Thymian}

bla bla

\section{Salbei}

bla bla

\section{Kamille}

bla bla

\section{Cistrose}

bla bla

\section{Beifuß}

bla bla

\section{Ysop}

bla bla

\section{Giersch}

bla bla

\section{Bärlauch}

bla bla

\section{Lavendel}

bla bla

\section{Beiwell}

bla bla

\section{Malvenblüten}

bla bla

\section{Rosenblüten}

bla bla

\section{Lindenblüten}

bla bla

\section{Gänseblümchen}

bla bla

\section{Anis}

bla bla

\section{Kümmel}

bla bla

\section{Liebstöckel}

bla bla

\section{Holunder}

bla bla

\section{Himbeerblätter}

bla bla

\section{Erdbeerblätter}

bla bla

\section{Johannisbeerblätter}

bla bla

\section{Eichenrinde}

bla bla

\section{Schwarztee}

bla bla

\section{Angelikawurzel}

bla bla

\section{Perlagonienwurzel}

bla bla

\section{Ingwer}

bla bla

\section{Fenchelsamen}

bla bla

\section{Chili}

bla bla

\section{Honig}

bla bla

\section{Propolis}

bla bla

\section{Olivenöl}

bla bla


