
%% ==============================
\chapter{Kräuterkunde}
%% ==============================
\label{   }
%% ==============================


%% ==============================
\section{Vorwort}
%% ==============================
\label{   }
%% ==============================


In diesem Kapitel werden die Wirkung und die Wirkstoffe der einzelnen Kräuter und weiterer Pflanzenteile erklärt. Dies wird hier zentral gemacht, als zentrales Nachschlagewerk, damit nicht bei jedem Kräuterauszug die Wirkung erneut aufgezählt werden muss. Dennoch schließt das eine einzelne Auszählung nicht aus, besonders um den Auszug der Seiten einzeln eingenständig weitergeben zu können. Hierbei sollen die Aussagen mit Literaturquellen belegt werden. Es können wörtliche Übernehmungen vorkommen.

% TODO Wirkstoffe alphabetisch sortieren
% TODO Wirkungsweisen alphabetisch sortieren
% TODO Literaturnachweise alphabetisch sortieren
% TODO sections alphabetisch sortieren
% TODO Wermut, Senf, Wacholder, Hanf, Stechapfel, Steinklee, Rosskastanie



%% ==============================
\section{Schafgarbe (Achillea millefolium)}
%% ==============================
\label{   }
%% ==============================



bla bla text

\textbf{Wirkstoffe:} Azulene, Gerbstoffe, Flavonoide, ätherische Öle\\

\textbf{Wirkungsweisen:} entzündungshemmend, wundheilend, desinfizierend\\

Literaturnachweise: \cite{nedoma2018heilsalben}



%% ==============================
\section{Melisse}
%% ==============================
\label{   }
%% ==============================


melisse
beruhigend auf das nervensystem




%% ==============================
\section{Aloe Vera}
%% ==============================
\label{   }
%% ==============================



bla bla



%% ==============================
\section{Goethepflanze}
%% ==============================
\label{   }
%% ==============================



bla bla



%% ==============================
\section{Kapuzinerkresse}
%% ==============================
\label{   }
%% ==============================



bla bla



%% ==============================
\section{Spitzwegerich}
%% ==============================
\label{   }
%% ==============================



bla bla


%% ==============================
\section{Breitwegerich}
%% ==============================
\label{   }
%% ==============================



bla bla


%% ==============================
\section{Frauenmantel}
%% ==============================
\label{   }
%% ==============================


frauenmantel
bei menstruationsbeschwerden


%% ==============================
\section{Ringelblume (Calendula officinalis)}
%% ==============================
\label{   }
%% ==============================

ringelblume
wundheilend

\textbf{Wirkstoffe:} Carotinoide, Flavonoide, Querecetin, Saponine, Salicylsäure\\

\textbf{Wirkungsweisen:} wundheilend, hautregenerierend, zusammenziehend, antibakteriell, entzündungshemmend, erweichend \\

Literaturnachweise: \cite{nedoma2018heilsalben}



%% ==============================
\section{Minze}
%% ==============================
\label{   }
%% ==============================



bla bla



%% ==============================
\section{Johanniskraut (Hypericum perforatum)}
%% ==============================
\label{   }
%% ==============================



Ein Kraut voller Kraft, man sagt, Johanniskraut speichert die Kraft der Sonne.

\textbf{Wirkstoffe:} Hypericin, Hyperforin, Flavonoide\\

\textbf{Wirkungsweisen:} zellschützend, antioxidativ, krebshemmend, antibakteriell, angstlösend, stimmunsaufhellend, lichtsensibilisierend, antidepressiv\\

Literaturnachweise: \cite{nedoma2018heilsalben}

Johanniskraut
Neurodermitis
antibakteriell
entzündungshemmend
wundheilend
beruhigend als tee

%% ==============================
\section{Thymian}
%% ==============================
\label{   }
%% ==============================


thymian
schleimlösend
gegen husten


%% ==============================
\section{Salbei}
%% ==============================
\label{   }
%% ==============================



salbei	
entzündungen im mund und rachenraum
magen darm
übermäßige schweißbildung
bla bla


%% ==============================
\section{Kamille (Matricaria chamomilla)}
%% ==============================
\label{   }
%% ==============================


kamille
magen darm
ätherische öle
anti entzündlich
antibakteriell
hautprobleme
entzündungen im hals
beruhigend auf das nervensystem

\textbf{Wirkstoffe:} Chamazulen, Cumarine, Flavonoide, ätherische Öle\\

\textbf{Wirkungsweisen:} antibakteriell, wundheilend, beruhigend\\

Literaturnachweise: \cite{nedoma2018heilsalben}


%% ==============================
\section{Cistrose}
%% ==============================
\label{   }
%% ==============================

bla bla


%% ==============================
\section{Beifuß}
%% ==============================
\label{   }
%% ==============================

bla bla



%% ==============================
\section{Ysop}
%% ==============================
\label{   }
%% ==============================

bla bla


%% ==============================
\section{Giersch}
%% ==============================
\label{   }
%% ==============================

bla bla


%% ==============================
\section{Bärlauch}
%% ==============================
\label{   }
%% ==============================

bla bla


%% ==============================
\section{Lavendel}
%% ==============================
\label{   }
%% ==============================

lavendel	
beruhigend
ernten bei trockenheit
gegen angstzustände
gegen innere unruhe
gegen zahn und kopfschmerzen

%% ==============================
\section{Beiwell (Symphytum officinalis)}
%% ==============================
\label{   }
%% ==============================

bla bla

\textbf{Wirkstoffe:} Allantoin, Cholin, Schleimstoffe, Gerbstoffe, Flavonoide, Kieselsäure, Alkaloide\\

\textbf{Wirkungsweisen:} wundheilend, geweberegenerierend, abschwellend, schmerzlindernd, entzündungshemmend, erweichend, knochenwachstumsfördernd, blutstillend \\

Literaturnachweise: \cite{nedoma2018heilsalben}


%% ==============================
\section{Malvenblüten (Malva)}
%% ==============================
\label{   }
%% ==============================

bla bla

\textbf{Wirkstoffe:} Schleimstoffe, Anthocyane, Flavonoide, Gerbstoffe, Saponine\\

\textbf{Wirkungsweisen:} feuchtigkeitsspendend, hautschützend, entzündungshemmend, juckreizstillend, zusammenziehend \\

Literaturnachweise: \cite{nedoma2018heilsalben}


%% ==============================
\section{Rosenblüten (Rosa)}
%% ==============================
\label{   }
%% ==============================

bla bla

\textbf{Wirkstoffe:} Gerbstoffe, ätherische Öle\\

\textbf{Wirkungsweisen:} desinfizierend, zusammenziehend, entspannend, schmerzstillend, kühlend, abschwellend, entzündungshemmend, blutstillend \\

Literaturnachweise: \cite{nedoma2018heilsalben}


%% ==============================
\section{Lindenblüten (Tilia)}
%% ==============================
\label{   }
%% ==============================

bla bla

\textbf{Wirkstoffe:} Schleimstoffe, Flavonoide, Gerbstoffe, Saponine, Farnesol, ätherische Öle \\

\textbf{Wirkungsweisen:} beruhigend, wärmend, abschwellend, entspannend, krampflösend, entzündungshemmend \\

Literaturnachweise: \cite{nedoma2018heilsalben}


%% ==============================
\section{Gänseblümchen (Bellisperennis)}
%% ==============================
\label{   }
%% ==============================


gänseblümchen	
tee erkältungen
öl prellungen, entzündungen, unten herum



\textbf{Wirkstoffe:} Bitterstoffe, Saponin, Schleimstoffe\\

\textbf{Wirkungsweisen:} schmerzstillend, wundheilend, hautklärend\\

Literaturnachweise: \cite{nedoma2018heilsalben}


%% ==============================
\section{Anis}
%% ==============================
\label{   }
%% ==============================

bla bla


%% ==============================
\section{Kümmel}
%% ==============================
\label{   }
%% ==============================

bla bla


%% ==============================
\section{Liebstöckel}
%% ==============================
\label{   }
%% ==============================

bla bla


%% ==============================
\section{Holunder}
%% ==============================
\label{   }
%% ==============================

bla bla



%% ==============================
\section{Himbeerblätter}
%% ==============================
\label{   }
%% ==============================

bla bla



%% ==============================
\section{Erdbeerblätter}
%% ==============================
\label{   }
%% ==============================

bla bla



%% ==============================
\section{Johannisbeerblätter}
%% ==============================
\label{   }
%% ==============================

bla bla



%% ==============================
\section{Eichenrinde}
%% ==============================
\label{   }
%% ==============================

bla bla



%% ==============================
\section{Schwarztee}
%% ==============================
\label{   }
%% ==============================


Schwarztee
Neurodermitis
bitterstoffe


%% ==============================
\section{Angelikawurzel}
%% ==============================
\label{   }
%% ==============================

bla bla



%% ==============================
\section{Perlagonienwurzel}
%% ==============================
\label{   }
%% ==============================

bla bla



%% ==============================
\section{Ingwer}
%% ==============================
\label{   }
%% ==============================

bla bla



%% ==============================
\section{Fenchelsamen}
%% ==============================
\label{   }
%% ==============================

bla bla



%% ==============================
\section{Chili}
%% ==============================
\label{   }
%% ==============================

bla bla



%% ==============================
\section{Honig}
%% ==============================
\label{   }
%% ==============================

bla bla



%% ==============================
\section{Propolis}
%% ==============================
\label{   }
%% ==============================

bla bla



%% ==============================
\section{Olivenöl}
%% ==============================
\label{   }
%% ==============================

bla bla



%% ==============================
\section{Majoran}
%% ==============================
\label{   }
%% ==============================

majoran
magen darm
krampfartiger husten


%% ==============================
\section{Hopfen}
%% ==============================
\label{   }
%% ==============================

hopfen
beruhigend
einschlaffördernd
durchschlaffördernd
bitterstoffe->beruhigend




%% ==============================
\section{Arnika}
%% ==============================
\label{   }
%% ==============================


	
arnika
wundheilend



%% ==============================
\section{Kickerhähnle}
%% ==============================
\label{   }
%% ==============================


kickerhähnle
leckere blüten




%% ==============================
\section{Bärwurz}
%% ==============================
\label{   }
%% ==============================

bärwurz
potenzsteigerung
aphrodisierend



%% ==============================
\section{Süßholz}
%% ==============================
\label{   }
%% ==============================


süßholz
neurodermitis



