\chapter{Tinkturen}

\section{Vorwort}

% TODO Einleitung

\lipsum[1-5]
\newpage




\section{Chillitinktur}

Chillis sind der Inbegriff der Schärfe. Ihr Wirkstoff Capsaicin reizt die Schleimhäute und Geschmacksknospen schmerzhaft, was wir als scharfen Geschmack wahrnehmen.\\
In relativ hoher Konzentration wird auch normale Haut durch das Capsaicin gereizt und die Durchblutung gefördert, weshalb man Chilis äusserlich gegen Gelenkschmerzen, Hexenschuss und andere Probleme des Bewegungsapparat anwenden kann.\\
Das Capsaicin löst sich besonders gut in Alkohol. Daher bietet es sich an, eine Tinktur aus Chillis herzustellen.\\
Wegen der intensiven Schärfe sollte man so eine Tinktur aber nur verdünnt und nur mit wenigen Tropfen Chilli-Tinktur innerlich anwenden. Am besten probiert man es zunächst mit einem einzelnen Tropfen auf einen Esslöffel Wasser aus, und steigert dann allmählich, bis es scharf genug ist.\\
Je nach Chilli-Sorte kann eine Chilli-Tinktur sehr unterschiedlich scharf sein. Es gibt extrem scharfe Sorten, wie beispielsweise die berühmten Habanero oder auch die Jalapeno aus Mexiko. Kleine, spitze Chillis im normalen Handel kommen häufig aus Thailand und sind im vergleich zu den superscharfen Sorten zwar sehr mild, aber für europäische Gaumen dennoch kräftig scharf.\\
Die etwas grösseren Brüder der Chilis, die Pepperoni, wie sie in Italien genannt werden, sind vergleichsweise mild, aber häufig immer noch so scharf, dass einem beim Essen das Wasser in die Augen schiesst.\\

% TODO Quellen
\cite{tinkturen}  ~\cite{heilkraeuterlexikon}

\url{https://www.chemie.de/lexikon/Capsaicin.html}

\index{Gelenkschmerzen} \index{Ischias} \index{Hexenschuss} \index{Zerrungen} \index{Verspannungen} \index{Nackenschmerzen} \index{Rückenschmerzen} \index{Handgelenksschmerzen} \index{Muskelkater} \index{Muskelschmerzen} \index{Kreislaufanregung} \index{Magenschwäche} \index{Rheuma} \index{Blähungen} \index{Verdauungsschwäche} \index{Appetitlosigkeit} \index{Appetitlosigkeit}

\subsection{Anwendung}

Eine Chillitinktur kann sehr vielseitig verwendet werden, wobei sie entweder innerlich verdünnt 2 bis 3  mal täglich 2 bis 10 Tropfen eingenommen oder äußerlich aufgetupft wird. Bei innerlicher Anwendung muss der Zustand des Magens und der Schärfegrad der Tinktur berücksichtigt werden, da Schärfe ebenso dem Magen schädigen und eine Gastritis herbeirufen kann. 

\begin{itemize}
	\item Appetitlosigkeit
	\item Verdauungsschwäche
	\item Blähungen
	\item Magenschwäche
	\item Rheuma
	\item Kreislaufanregung
	\item Muskelschmerzen
	\item Muskelkater
	\item Rücken- \& Nackenschmerzen
	\item Verspannungen
	\item Zerrungen
	\item Hexenschuss
	\item Ischias
	\item Gelenkschmerzen 
\end{itemize}

\subsection{Herstellung}

Die Chilifrüchte werden zerkleinert und mit den Kernen in ein Schraubglas gegeben. Dieses wird mit mind. 40\%igem Alkohol aufgefüllt, verschlossen und an einen dunklen, warmen Ort deponiert. Täglich schütteln, nach ca 2 Monaten kann die Tinktur abfiltriert und abgefüllt werden.

%\subsection{Anmerkungen}



\newpage



\section{Johanniskrauttinktur}

Konzentrierte Version des Rotöls als Alkoholauszug. Prinzipiell hat die Tinktur die gleiche Wirkung wie das Öl, nur dass damit nicht massiert werden kann. Dazu ist die längere Anwendung auf der Haut abzuwägen, da sowohl Alkohol als auch Johanniskraut die Haut austrocknen. Die Tinktur kann auch zu einer Johanniskraut Salbe bei ihrer Herstellung hinzugegeben werden, um den Wirkstoffgehalt zu erhöhen.

% TODO Quellen
\cite{tinkturen} \cite{nedoma2018heiltinkturen}

\url{https://tinkturen-selbstgemacht.de/rezepte/johanniskraut-tinktur.htm}

\index{Winterdepression} \index{Lichtmangel} \index{Depressive Stimmung} \index{Schlaflosigkeit} \index{Angstzustände} \index{Blutergüsse}
\index{Entzündungen} \index{Prellungen} \index{Wunden} \index{Juckreiz} \index{Sonnenbrand} \index{Schmerzen} \index{Schwellungen} \index{Wundheilung} \index{Wundversorgung} \index{Massage} \index{Muskelentspannung} \index{Winterdepression} \index{Antidepressivum} \index{Narbenentfernung} \index{Narben} \index{Verspannungen}

\subsection{Anwendung}

\textbf{Innerliche Anwendung:} 10 Tropfen der Tinktur dreimal am Tag eine halbe Stunde vor den Mahlzeiten einnehmen. Die Kur im Spätherbst beginne und 8 Wochen durchführen, anschließend 1 Woche pausieren. Die Wirkung der Tinktur kann durch Aufenthalt in der Natur sowie durch die Einnahme von Carotinen und Vitamin D intensiviert werden.\\

\textbf{Äußerliche Anwendung:} Die Tinktur 1:3 mit Wasser verdünnen und in eine Zerstäuberflasche füllen. Die betroffenen Stellen besprühen, betupfen oder eine Kompresse auflegen.\\

\subsection{Herstellung}

Johanniskrautblüten oder geschreddertes und getrocknetes Kraut in ein Glas geben und mit Alkohol >40\% übergießen. Fest verschließen und einige Wochen bis Monate dunkel extrahieren lassen. Ab und zu schütteln.

\subsection{Anmerkungen}

Nicht anwenden bei gleichzeitiger Einnahme von Herzmedikamenten! Johanniskraut innerlich aktiviert das Enzym Cytochrom P450 in der Leber. Dadurch werden bestimmte Hormone und Medikamente schneller abgebaut. Auswirkungen davon sind unter Anderem die reduzierte Wirkung von hormonellen Verhültungsmitteln, aber auch die zumindest reduzierte Wirkung bestimmter AIDS-Medikamente, Immunsuppressiva sowie anderen Medikamenten wie z.B. Diazepam, Ciclosporin und Amiodaron. Dazu wird die Wirkung der Anti-Baby-Pille aufgehoben.



\newpage



\section{Tinktur für die Nerven}
% TODO Einleitung

\cite{nedoma2018heiltinkturen}

\index{Schlafstörungen} \index{Beruhigung} \index{Nervenberuhigung} \index{Stress} \index{Angstzustände} \index{Depressive Stimmung} \index{Akne} \index{Sonnenbrand} \index{Fieber} \index{Kopfschmerzen}

\subsection{Anwendung}

Die Tinktur beruhigt das Nervensysten, hilft bei Stress und Anstzuständnen und diese abzubauen, normalisiert Schlafstörungen und hellt depressive Verstimmungen auf. Äußerlich  kann sie bei Akne, Sonnenbrand, Kopfschmerzen und Fieber helfen.

\textbf{Innerliche Anwendung:} Bei Bedarf 10 Tropfen pur einnehmen oder 30 Tropfen mit 500ml Lindenblütentee vermischen und über den Tag verteilt trinken. Die Kur drei Wochen durchführen, anschließend eine Woche pausieren.\\ 

\textbf{Äußerliche Anwendung:} Als Pflaster, Kompresse oder Einreibung wird die Tinktur 1:1 mit Wasser oder Tee verdünnt und auf die betroffenen Stellen aufgetragen.\\

\subsection{Herstellung}

\begin{itemize}
	\item 2 EL getrocknete Rosenblüten
	\item 2 El getrocknete Lindenblüten
	\item 200 ml 40\%-iger Alkohol
\end{itemize}

Die Blüten in ein Glas geben und sie mit dem Alkohol übergießen. Fest verschließen und ab und zu schütteln. Mindestens 4 Wochen an einem dunklen, warmen Ort extrahieren lassen. Danach abfiltrieren und dunkel aufbewahren. Die Tinktur ist 1 Jahr haltbar.

%\subsection{Anmerkungen}



\newpage



\section{Schafgarbentinktur}

% TODO Einleitung, Quellen

\index{Erkältung} \index{Halsschmerzen} \index{Magenbeschwerden} \index{Erkältung}

\subsection{Anwendung}
% TODO
\textbf{Erkältung:} \\ 

\textbf{Halsschmerzen:} \\ 

\textbf{Magenbeschwerden:} \\ 

\textbf{Erkältung:} \\ 

\subsection{Herstellung}
% TODO

%\subsection{Anmerkungen}



\newpage


\section{Ingwertinktur}

% TODO Einleitung, Quellen

\url{https://www.celticgarden.de/ingwertinktur-gegen-erkaeltungen-und-magenbeschwerden/}

\index{Erkältungskrankheiten} \index{Übelkeit} \index{Magenbeschwerden} \index{Kopfschmerzen}

\subsection{Anwendung}
% TODO
\textbf{Kopfschmerzen:} \\ 

\textbf{Magenbeschwerden:} \\ 

\textbf{Übelkeit:} \\ 

\textbf{Erkältung:} \\ 


\subsection{Herstellung}

Ingwer mit Schale in kleine Stücke schneiden und in ein Schraubglas geben. Das Glas mit reinem Alkohol füllen, zB mit Primasprit, Vodka oder Korn mit mindestens 40\% Alkoholgehalt. Mindestens die Ingwerstücke müssen vom Alkohol bedeckt sein. Dann an einem dunklen, warmen Ort stehen lassen, täglich schütteln. Nach 2 Monaten kann die Tinktur abfiltriert werden. Die Ingwerstückchen nochmals auspressen und alles wiederholt filtrieren.

%\subsection{Anmerkungen}



\newpage



\section{Propolistinktur}

% TODO Einleitung, Quellen

\index{Mandelentzündung} \index{Erkältung} \index{Verletzungen im Mund}

\subsection{Anwendung}
% TODO
\textbf{Mandelentzündung:} \\

\textbf{Erkältung:} \\ 

\textbf{Verletzungen im Mund:} \\

\subsection{Herstellung}
% TODO

%\subsection{Anmerkungen}



\newpage



\section{Ringelblumentinktur}

% TODO Einleitung, Quellen, Index

\cite{tinkturen}

\url{https://tinkturen-selbstgemacht.de/rezepte/ringelblumen-tinktur.htm}

\index{Menstruationsbeschwerden}

\subsection{Anwendung} 
% TODO
\subsection{Herstellung}
% TODO

%\subsection{Anmerkungen}



\newpage



\section{Kamillentinktur}

% TODO Einleitung, Quellen, Index

\cite{tinkturen}

\url{https://tinkturen-selbstgemacht.de/rezepte/kamillen-tinktur.htm}

\subsection{Anwendung}
% TODO
\subsection{Herstellung}
% TODO

%\subsection{Anmerkungen}



\newpage



\section{Blähungstinktur}

% TODO Einleitung, Quellen, Index

\cite{tinkturen}  

\url{https://tinkturen-selbstgemacht.de/rezepte/blaehungs-tinktur.htm}

\subsection{Anwendung}
% TODO

\subsection{Herstellung}
% TODO
\begin{itemize}
	\item Fenchel
	\item Anis
	\item Kümmel
	\item Angelikawurzel
	\item Liebstöckelwurzel
	\item Pefferminzblätter
	\item Kamillenblüten
\end{itemize}

%\subsection{Anmerkungen}



\newpage



\section{Hustentinktur}

% TODO Einleitung, Quellen

\cite{tinkturen}  

\url{https://tinkturen-selbstgemacht.de/rezepte/husten-tinktur.htm} 

\index{Erkältungskrankheiten} \index{Husten}

\subsection{Anwendung}
% TODO
\subsection{Herstellung}
% TODO
\begin{itemize}
	\item Thymian
	\item Ysop
	\item Salbei
\end{itemize}

%\subsection{Anmerkungen}



\newpage



\section{Gierschtinktur}

% TODO Einleitung, Quellen

\index{Rheuma} \index{Gicht} \index{Arthrose} \index{Blasenentzündung} \index{Ischiasbeschwerden}

\subsection{Anwendung}
% TODO
\subsection{Herstellung}
% TODO

%\subsection{Anmerkungen}



\newpage



\section{Bärlauchtinktur}

% TODO Einleitung, Quellen, Index

\index{Darmregeneration}

\subsection{Anwendung}
% TODO
\textbf{Darmregeneration:} \\ 

\subsection{Herstellung}
% TODO

%\subsection{Anmerkungen}



\newpage



\section{Perlagonientinktur}

% TODO Einleitung, Quellen

\index{Erkältungskrankheiten} \index{Infekte}

\subsection{Anwendung}
% TODO
\textbf{Infekte:} \\ 

\textbf{Erkältungskrankheiten:} \\ 


\subsection{Herstellung}
% TODO

\subsection{Anmerkungen}

Käuflich erwerbbare Präparate sind z.B. Umckaloabo oder Kapla Pelargo (Aldi).







