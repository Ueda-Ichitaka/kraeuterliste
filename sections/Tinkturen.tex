\chapter{Tinkturen}

\section{Vorwort}

\lipsum[1-5]
\newpage


\section{Scharfgabentinktur}

\index{Erkältung} \index{Halsschmerzen} \index{Magenbeschwerden} \index{Erkältung}

\subsection{Anwendung}

\textbf{Erkältung:} \\ 

\textbf{Halsschmerzen:} \\ 

\textbf{Magenbeschwerden:} \\ 

\textbf{Erkältung:} \\ 

\subsection{Herstellung}

\subsection{Anmerkungen}






\section{Ingwertinktur}

\url{https://www.celticgarden.de/ingwertinktur-gegen-erkaeltungen-und-magenbeschwerden/}

\index{Erkältungskrankheiten} \index{Übelkeit} \index{Magenbeschwerden} \index{Kopfschmerzen}


\subsection{Anwendung}

\textbf{Kopfschmerzen:} \\ 

\textbf{Magenbeschwerden:} \\ 

\textbf{Übelkeit:} \\ 

\textbf{Erkältung:} \\ 


Das kleine Buch Heiltinkturen aus Wald und Wiese


\subsection{Herstellung}

Ingwer mit Schale in kleine Stücke schneiden und in ein Schraubglas geben. Das Glas mit reinem Alkohol füllen, zB mit Primasprit, Vodka oder Korn mit mindestens 40\% Alkoholgehalt. Mindestens die Ingwerstücke müssen vom Alkohol bedeckt sein. Dann an einem dunklen, warmen Ort stehen lassen, täglich schütteln. Nach 2 Monaten kann die Tinktur abfiltriert werden. Die Ingwerstückchen nochmals auspressen und alles wiederholt filtrieren.

\subsection{Anmerkungen}






\section{Gierschtinktur}

\index{Rheuma} \index{Gicht} \index{Arthrose} \index{Blasenentzündung} \index{Ischiasbeschwerden}

\subsection{Anwendung}

\subsection{Herstellung}

\subsection{Anmerkungen}







\section{Bärlauchtinktur}

\index{Darmregeneration}

\subsection{Anwendung}

\textbf{Darmregeneration:} \\ 

\subsection{Herstellung}

\subsection{Anmerkungen}






\section{Perlagonientinktur}


\index{Erkältungskrankheiten} \index{Infekte}

\subsection{Anwendung}


\textbf{Infekte:} \\ 

\textbf{Erkältungskrankheiten:} \\ 


\subsection{Herstellung}

\subsection{Anmerkungen}

Käuflich erwerbbare Präparate sind z.B. Umckaloabo oder Kapla Pelargo (Aldi).






\section{Propolistinktur}

\index{Mandelentzündung} \index{Erkältung} \index{Verletzungen im Mund}

\subsection{Anwendung}

\textbf{Mandelentzündung:} \\

\textbf{Erkältung:} \\ 

\textbf{Verletzungen im Mund:} \\


\subsection{Herstellung}

\subsection{Anmerkungen}






\section{Chillitinktur}

Chillis sind der Inbegriff der Schärfe. Ihr Wirkstoff Capsaicin reizt die Schleimhäute und Geschmacksknospen schmerzhaft, was wir als scharfen Geschmack wahrnehmen.

In relativ hoher Konzentration wird auch normale Haut durch das Capsaicin gereizt und die Durchblutung gefördert, weshalb man Chilis äusserlich gegen Gelenkschmerzen, Hexenschuss und andere Probleme des Bewegungsapparat anwenden kann.

Das Capsaicin löst sich besonders gut in Alkohol. Daher bietet es sich an, eine Tinktur aus Chillis herzustellen.

Wegen der intensiven Schärfe sollte man so eine Tinktur aber nur verdünnt und nur mit wenigen Tropfen Chilli-Tinktur innerlich anwenden. Am besten probiert man es zunächst mit einem einzelnen Tropfen auf einen Esslöffel Wasser aus, und steigert dann allmählich, bis es scharf genug ist.

Je nach Chilli-Sorte kann eine Chilli-Tinktur sehr unterschiedlich scharf sein. Es gibt extrem scharfe Sorten, wie beispielsweise die berühmten Habanero oder auch die Jalapeno aus Mexiko. Kleine, spitze Chillis im normalen Handel kommen häufig aus Thailand und sind im vergleich zu den superscharfen Sorten zwar sehr mild, aber für europäische Gaumen dennoch kräftig scharf.

Die etwas grösseren Brüder der Chilis, die Pepperoni, wie sie in Italien genannt werden, sind vergleichsweise mild, aber häufig immer noch so scharf, dass einem beim Essen das Wasser in die Augen schiesst.

\cite{tinkturen}  ~\cite{heilkraeuterlexikon}

\url{https://www.chemie.de/lexikon/Capsaicin.html}


\index{Gelenkschmerzen} \index{Ischias} \index{Hexenschuss} \index{Zerrungen} \index{Verspannungen} \index{Nackenschmerzen} \index{Rückenschmerzen} \index{Muskelkater} \index{Muskelschmerzen} \index{Kreislaufanregung} \index{Magenschwäche} \index{Rheuma} \index{Blähungen} \index{Verdauungsschwäche} \index{Appetitlosigkeit} \index{Appetitlosigkeit}


\subsection{Anwendung}

Eine Chillitinktur kann sehr vielseitig verwendet werden, wobei sie entweder innerlich verdünnt 2 bis 3  mal täglich 2 bis 10 Tropfen eingenommen oder äußerlich aufgetupft wird. Bei innerlicher Anwendung muss der Zustand des Magens und der Schärfegrad der Tinktur berücksichtigt werden, da Schärfe ebenso dem Magen schädigen und eine Gastritis herbeirufen kann. 



\begin{itemize}
	\item Appetitlosigkeit
	\item Verdauungsschwäche
	\item Blähungen
	\item Magenschwäche
	\item Rheuma
	\item Kreislaufanregung
	\item Muskelschmerzen
	\item Muskelkater
	\item Rücken- \& Nackenschmerzen
	\item Verspannungen
	\item Zerrungen
	\item Hexenschuss
	\item Ischias
	\item Gelenkschmerzen 
\end{itemize}

\subsection{Herstellung}

Die Chilifrüchte werden zerkleinert und mit den Kernen in ein Schraubglas gegeben. Dieses wird mit mind. 40\%igem Alkohol aufgefüllt, verschlossen und an einen dunklen, warmen Ort deponiert. Täglich schütteln, nach ca 2 Monaten kann die Tinktur abfiltriert und abgefüllt werden.

\subsection{Anmerkungen}







\section{Johanniskrauttinktur}

\cite{tinkturen} \cite{nedoma2018heiltinkturen}

\url{https://tinkturen-selbstgemacht.de/rezepte/johanniskraut-tinktur.htm}

\index{Winterdepression} \index{Lichtmangel} \index{Depressive Stimmung} \index{Schlaflosigkeit} \index{Angstzustände} \index{Blutergüsse}
\index{Entzündungen} \index{Prellungen} \index{Wunden} \index{Juckreiz} \index{Sonnenbrand} \index{Schmerzen} \index{Schwellungen} 

\subsection{Anwendung}

\subsection{Herstellung}

\subsection{Anmerkungen}

Nicht anwenden bei gleichzeitiger Einnahme von Herzmedikamenten!





\section{Ringelblumentinktur}

\cite{tinkturen}

\url{https://tinkturen-selbstgemacht.de/rezepte/ringelblumen-tinktur.htm}

\index{Menstruationsbeschwerden}

\subsection{Anwendung} 

\subsection{Herstellung}

\subsection{Anmerkungen}






\section{Kamillentinktur}

\cite{tinkturen}

\url{https://tinkturen-selbstgemacht.de/rezepte/kamillen-tinktur.htm}

\subsection{Anwendung}

\subsection{Herstellung}

\subsection{Anmerkungen}







\section{Blähungstinktur}

\cite{tinkturen}  

\url{https://tinkturen-selbstgemacht.de/rezepte/blaehungs-tinktur.htm}

\subsection{Anwendung}


\subsection{Herstellung}


\begin{itemize}
	\item Fenchel
	\item Anis
	\item Kümmel
	\item Angelikawurzel
	\item Liebstöckelwurzel
	\item Pefferminzblätter
	\item Kamillenblüten
\end{itemize}

\subsection{Anmerkungen}




\section{Hustentinktur}

\cite{tinkturen}  

\url{https://tinkturen-selbstgemacht.de/rezepte/husten-tinktur.htm} 


\index{Erkältungskrankheiten} \index{Husten}

\subsection{Anwendung}

\subsection{Herstellung}

\begin{itemize}
	\item Thymian
	\item Ysop
	\item Salbei
\end{itemize}

\subsection{Anmerkungen}










