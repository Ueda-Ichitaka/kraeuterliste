\chapter{Tinkturen}

\section{Vorwort}

\lipsum[1-5]
\newpage


\section{Scharfgabentinktur}

\subsection{Anwendung}

\textbf{Erkältung:} \\ \\
\textbf{Halsschmerzen:} \\ \\
\textbf{Magenbeschwerden:} \\ \\
\textbf{Erkältung:} \\ \\


\subsection{Herstellung}

\subsection{Anmerkungen}



\section{Ingwertinktur}

\subsection{Anwendung}

\textbf{Kopfschmerzen:} \\ \\
\textbf{Magenbeschwerden:} \\ \\
\textbf{Übelkeit:} \\ \\
\textbf{Erkältung:} \\ \\

\cite{celticingwer}
Das kleine Buch Heiltinkturen aus Wald und Wiese


\subsection{Herstellung}

Ingwer mit Schale in kleine Stücke schneiden und in ein Schraubglas geben. Das Glas mit reinem Alkohol füllen, zB mit Primasprit, Vodka oder Korn mit mindestens 40\% Alkoholgehalt. Mindestens die Ingwerstücke müssen vom Alkohol bedeckt sein. Dann an einem dunklen, warmen Ort stehen lassen, täglich schütteln. Nach 2 Monaten kann die Tinktur abfiltriert werden. Die Ingwerstückchen nochmals auspressen und alles wiederholt filtrieren.

\subsection{Anmerkungen}




\section{Gierschtinktur}

\subsection{Anwendung}

\subsection{Herstellung}

\subsection{Anmerkungen}





\section{Bärlauchtinktur}

\subsection{Anwendung}

\textbf{Darmregeneration:} \\ \\

\subsection{Herstellung}

\subsection{Anmerkungen}





\section{Perlagonientinktur}

\subsection{Anwendung}

\textbf{Infekte:} \\ \\
\textbf{Erkältungskrankheiten:} \\ \\


\subsection{Herstellung}

\subsection{Anmerkungen}

Käuflich erwerbbare Präparate sind z.B. Umckaloabo oder Kapla Pelargo (Aldi).



\section{Propolistinktur}

\subsection{Anwendung}

\textbf{Mandelentzündung:} \\ \\
\textbf{Erkältung:} \\ \\
\textbf{Verletzungen im Mund:} \\ \\


\subsection{Herstellung}

\subsection{Anmerkungen}

