\chapter{Tee}

\section{Vorwort}

Tee ist die wohl bekannteste Darreichungsform von Heilkräutern, auch wenn in den letzten Jahren seine Heilwirkung immer mehr in Vergessenheit geraten ist. Bei Erkältungskrankheiten hilft schon allein die Wärme und die konstante Flüssigkeitszufuhr. Aber Tee kann noch viel mehr, ein weckendes Morgenritual, Beruhigung von Geist, Magen-Darm Trakt, etc.\\

Generell kann man einen gewöhnlichen Tee als Aufguss bezeichnen, also als einen schwachen Auszug. Ein konzentrierterer Auszug ist ein Absud, dieser muss für längere Zeit ziehen.\\



\newpage



\section{Muttis guter Gartentee}

\subsection{Anwendung}

\subsection{Herstellung}

Für den Tee benötigt man folgende Zutaten:

\begin{itemize}
	\item Minze
	\item Melisse
	\item Himbeerblätter
	\item Erdbeerblätter
	\item Schwarze Johannisbeerblätter
\end{itemize}


\subsection{Anmerkungen}


\newpage


\section{Erkältungstee}

\index{Erkältungskrankheiten}

\subsection{Anwendung}

Bei \textbf{Erkältungskrankheiten} ein milder Tee, Zitrone liefert Vitamin C, der Honig Propolis und weitere heilende Honiginhalsstoffe, der Ingwer brennt die Krankheit von innen heraus, ohne dabei die Körpertemperatur zu erhöhen. Dieser Tee ist besonders geeignet für Personen, die Kräutertee geschmacklich abgeneigt sind.

\subsection{Herstellung}

Für den Tee benötigt man folgende Zutaten:

\begin{itemize}
	\item Thymian
	\item Salbei
	\item Ingwer
	\item evtl einfacher Kräutertee
	\item Zitronensaft
	\item Honig
\end{itemize}


\subsection{Anmerkungen}

Salbei wirkt abstillend, also nicht zu verwenden während der Stillzeit.


\newpage



\section{Ingwertee}

\url{https://www.praxis-kakizaki.de/2017/01/24/ingwer-wirkung-und-nebenwirkungen/}

\index{Erkältungskrankheiten} \index{Krankheiten aller Art} \index{Stärkung des Immunsystems}

\subsection{Anwendung}

Bei \textbf{Krankheiten aller Art} und zur \textbf{Stärkung des Immunsystems}.

\subsection{Herstellung}

Frischer Ingwer wird in Scheiben geschnitten, diese evtl noch weiter zerkleinert, und mit heißem Wasser übergossen. Auf eine Tasse 4-5 Scheibchen, bei einer Kanne mit Teesieb mehr. Zu beachten ist, dass der Ingwer mit zunehmender Ziehzeit scharf wird.

\subsection{Anmerkungen}


\newpage



\section{Schwarzteeabsud}

\index{Durchfallerkrankungen}

\subsection{Anwendung}

Ein Absud für \textbf{Durchfallerkrankungen}. Schwarztee enthält Bitter- und Gerbstoffe, die beruhigend und wie eine Schutzschicht auf die Magen- und Darmschleimhäute wirken, da diese sich zusammenziehen und damit weniger anfällig auf die auslösenden Bakterien und Viren sind. Je nach Schwere der Erkrankung wird der Absud unterschiedlich dosiert, bei Erbrechen sollte er stark verdünnt sein, ansonsten beliebig.

\subsection{Herstellung}

Schwarztee ca 20 - 30 Minuten ziehen lassen, je nach Schwere der Erkrankung unterschiedlich stark. Bei einfachem Durchfall kann ein Beutel auf eine große Tasse reichen, bei Erbrechen stark verdünnt, ein Beutel pro Kanne.

\subsection{Anmerkungen}


\newpage


\section{Cistrosentee}

\index{Erkältungskrankheiten}

\subsection{Anwendung}

\subsection{Herstellung}

\subsection{Anmerkungen}



\newpage


\section{Salbeitee}

\index{Erkältungskrankheiten} \index{Halsschmerzen} \index{Mandelentzündung}

\subsection{Anwendung}

\subsection{Herstellung}

\subsection{Anmerkungen}

Salbei wirkt abstillend, also nicht zu verwenden während der Stillzeit.


\newpage


\section{Kamillentee}

\index{Magenbeschwerden} \index{Magenschmerzen} \index{Gastritis}

\subsection{Anwendung}

\subsection{Herstellung}

\subsection{Anmerkungen}



\newpage


\section{Magentee}

\index{Magenbeschwerden} \index{Magenschmerzen} \index{Magen-Darm-Beschwerden}

\subsection{Anwendung}

\subsection{Herstellung}

In beliebiger Kombination

\begin{itemize}
	\item Kamille
	\item Minze
	\item Salbei
	\item Thymian
	\item Fenchel
	\item Anis
	\item Kümmel
	\item Melisse
\end{itemize}

\subsection{Anmerkungen}


\newpage


\section{Melissentee}

\index{Einschlafen} \index{Zur-Ruhe-kommen} \index{Angespanntheit} \index{Verdauungsprobleme} \index{Asthma} \index{Erkältungskrankheiten} \index{Kopfschmerzen} \index{Menstruationsbeschwerden} \index{Reizbarkeit} \index{Nervöse Herzschwäche} \index{Hypothereose} \index{Schilddrüsenüberfunktion}


\subsection{Anwendung}

Aufgrund ihres angenehmen Zitronenaromas und ihrer vielfältigen Wirkungen eignet sich die Melisse sehr gut zur Anwendung als Tee. Man kann sie sowohl als reinen Melissen-Tee trinken oder auch in Teemischungen, wo sie die Mischung geschmacklich aufwertet.
Abends hilft Melissentee, auf Wunsch mit Honig, beim \textbf{Zur-Ruhe-kommen} und \textbf{Einschlafen}.

Morgens oder bei \textbf{Angespanntheit} wirkt Melissentee hingegen belebend und erfrischend und gibt Stärke.

Das ist kein Widerspruch, denn Entspannung und Kräftigung passen durchaus zusammen.

Bei \textbf{Erkältung} hilft die Melisse die Atmung zu verbessern und eventuelles Fieber besser auszuhalten. Sogar bei \textbf{Asthma} kann Melissentee die Atmung erleichtern.

Melisse hilft auch bei der \textbf{Verdauung} und wirkt entkrampfend auf Magen und Darm. Nach längerer Krankheit kann sie den Appetit steigern.

Sie kann auch \textbf{Kopfschmerzen} und Reizungen des Nervensystems lindern, was für ihren Einsatz gegen \textbf{Neuralgien}\index{Neuralgien}, \textbf{Reizbarkeit} und \textbf{Unruhe}\index{Unruhe} spricht.

Auch bei \textbf{Menstruationsbeschwerden} kann die Melisse hilfreich sein. Sie entkrampft die Unterleibsorgane, sodass Schmerzen während der Periode schwinden.

In den \textbf{Wechseljahren}\index{Wechseljahren} kann die Melisse gegen viele der typischen Beschwerden helfen. Vor allem wenn man nachts nicht einschlafen kann, oder auch tagsüber von Unruhe oder schlechter Laune geplagt wird, hilft ein Melissentee wieder zur eigenen Mitte zu finden. Auch gegen die typischen Hitzewallungen und das Herzklopfen kann man die Melisse einsetzen.

Kommen wir zu einem besonders wichtigen Einsatzgebiet der Melisse: der \textbf{nervösen Herzschwäche}. Wenn man ärztlich abgeklärt hat, dass das Herz organisch gesund ist, kann man mithilfe der Melisse die Beschwerden lindern, die aufgrund innerer Unruhe oder anderen nerlichen Gründen bestehen können.

Bei einer \textbf{Hyperthyreose}, also einer \textbf{Schilddrüsenüberfunktion} ist sie anwendbar, nicht aber bei einer Hypothyreose, da Melisse die Aktivitäten einer überaktiven Schilddrüse normalisiert. Es reduziert die Produktion von Schilddrüsenhormonen.


\cite{heilkraeuterlexikon}  

\url{https://heilkraeuter.de/lexikon/melisse.htm}

\subsection{Herstellung}

\subsection{Anmerkungen}



\newpage



\section{Frauenmanteltee}

Zur Anwendung bei Periodenbeschwerden.

\index{Periodenschmerzen} \index{Menstruationsbeschwerden}

\subsection{Anwendung}

\subsection{Herstellung}

\subsection{Anmerkungen}



\newpage



\section{Innerer Heiltee}

\subsection{Anwendung}

\subsection{Herstellung}

\begin{itemize}
	\item Frauenmantel 
	\item Schafgarbe
	\item Beifuß
\end{itemize}

\subsection{Anmerkungen}



