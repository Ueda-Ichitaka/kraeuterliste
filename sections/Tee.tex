\chapter{Tee}

\section{Vorwort}

\lipsum[1-5]
\newpage



\subsection{Herstellung allgemein}



\section{Muttis guter Gartentee}

\subsection{Anwendung}

\subsection{Herstellung}

Für den Tee benötigt man folgende Zutaten:

\begin{itemize}
	\item Minze
	\item Melisse
	\item Himbeerblätter
	\item Erdbeerblätter
	\item Schwarze Johannisbeerblätter
\end{itemize}


\subsection{Anmerkungen}




\section{Erkältungstee}

\subsection{Anwendung}

Bei \textbf{Erkältungskrankheiten} ein milder Tee, Zitrone liefert Vitamin C, der Honig Propolis und weitere heilende Honiginhalsstoffe, der Ingwer brennt die Krankheit von innen heraus, ohne dabei die Körpertemperatur zu erhöhen. Dieser Tee ist besonders geeignet für Personen, die Kräutertee geschmacklich abgeneigt sind.

\subsection{Herstellung}

Für den Tee benötigt man folgende Zutaten:

\begin{itemize}
	\item Thymian
	\item Salbei
	\item Ingwer
	\item evtl einfacher Kräutertee
	\item Zitronensaft
	\item Honig
\end{itemize}


\subsection{Anmerkungen}



\section{Ingwertee}

\subsection{Anwendung}

Bei \textbf{Krankheiten aller Art} und zur \textbf{Stärkung des Immunsystems}.

\subsection{Herstellung}

Frischer Ingwer wird in Scheiben geschnitten, diese evtl noch weiter zerkleinert, und mit heißem Wasser übergossen. Auf eine Tasse 4-5 Scheibchen, bei einer Kanne mit Teesieb mehr. Zu beachten ist, dass der Ingwer mit zunehmender Ziehzeit scharf wird.

\subsection{Anmerkungen}

https://www.praxis-kakizaki.de/2017/01/24/ingwer-wirkung-und-nebenwirkungen/
ersetz mich zu text!




\section{Schwarzteeabsud}

\subsection{Anwendung}

Ein Absud für \textbf{Durchfallerkrankungen}. Schwarztee enthält Bitter- und Gerbstoffe, die beruhigend und wie eine Schutzschicht auf die Magen- und Darmschleimhäute wirken, da diese sich zusammenziehen und damit weniger anfällig auf die auslösenden Bakterien und Viren sind. Je nach Schwere der Erkrankung wird der Absud unterschiedlich dosiert, bei Erbrechen sollte er stark verdünnt sein, ansonsten beliebig.

\subsection{Herstellung}

Schwarztee ca 20 - 30 Minuten ziehen lassen, je nach Schwere der Erkrankung unterschiedlich stark. Bei einfachem Durchfall kann ein Beutel auf eine große Tasse reichen, bei Erbrechen stark verdünnt, ein Beutel pro Kanne.

\subsection{Anmerkungen}





\section{Cistrosentee}

\subsection{Anwendung}

\subsection{Herstellung}

\subsection{Anmerkungen}




\section{Salbeitee}

\subsection{Anwendung}

\subsection{Herstellung}

\subsection{Anmerkungen}





\section{Kamillentee}

\subsection{Anwendung}

\subsection{Herstellung}

\subsection{Anmerkungen}






\section{Magentee}

\subsection{Anwendung}

\subsection{Herstellung}

In beliebiger Kombination

\begin{itemize}
	\item Kamille
	\item Minze
	\item Salbei
	\item Thymian
	\item Fenchel
	\item Anis
	\item Kümmel
\end{itemize}

\subsection{Anmerkungen}





\section{Schilddrüsentee}

\subsection{Anwendung}

\subsection{Herstellung}

Melisse

\subsection{Anmerkungen}

Bei einer Hyperthyreose anwendbar, nicht Hypothyreose, da Melisse die Aktivitäten einer überaktiven Schilddrüse normalisiert. Es reduziert die Produktion von Schilddrüsenhormonen.






\section{Frauenmanteltee}

\subsection{Anwendung}

\subsection{Herstellung}

\subsection{Anmerkungen}


