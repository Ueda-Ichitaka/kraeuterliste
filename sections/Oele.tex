\chapter{Öle}

\section{Vorwort}

\lipsum[1-5]
\newpage





\subsection{Allgemeine Anmerkungen}





\section{Johanniskrautöl}

Auch Rotöl oder Blut genannt. Ein sehr potentes Heilmittel

\subsection{Anwendung}

\textbf{Wundheilung:} Bei oberflächlichen Wunden wie z.B. Schürfwunden die Wunde reinigen, eine ES-Kompresse mit dem Öl tränken, auf die Wunde aufbringen und fixieren. Nach ca 5 Stunden den Verband wechseln. Fortfahren nach den üblichen Regeln der Wundversorgung. Bei infizierten Wunden mit Povidiod behandeln oder einen Arzt aufsuchen.\\ \\
\textbf{Muskelentspannung:} Das Öl auf die betroffenen Körperregionen aufbringen und einmassieren, am besten mit einer Massage. \\ \\
\textbf{Winterdepression:} Täglich etwas Öl auf der Haut an beliebigen Körperstellen wie zB. Arme, Schultern, Rücken verteilen und einmassieren. Das Johanniskraut bewirkt die Produktion von Vitamin D in der Haut. Wichtig, die entsprechenden Körperteile nicht direkter Sonne aussetzen, da eine stark Erhöhte Verbrennungsgefahr besteht! \\ \\
\textbf{Antidepressivum:} Täglich 10-20 Tropfen zusammen mit Tee oder einem anderen Getränk einnehmen. \\ \\
\textbf{Narbenentfernung:} Morgens und Abends die Narbe im Wechsel mit einer Ringelblumensalbe und Rotöl einreiben, dies über einen längeren Zeitraum. Rotöl tendiert dazu, die Haut auszutrocknen, daher die Ringelblumensalbe.

\subsection{Herstellung}

Frisches Johanniskraut ist am 21. Juni zu sammeln. Die Pflanze enthält viele kleine rote Punktre, das ist der Wirkstoff. Man nehme promär die Blüten, entferne also das Kraut untenrum grob, trockne diese ein paar Stunden leicht an und gebe sie in eine Karaffe mit Korken. Die Karaffe wird nun mit Öl, idealerweise Olivenöl aufgefüllt, der Korken in Frischhaltefolie eingewickelt, und das ganze an einem warmen, sonnigen Ort deponiert. Jeden Tag schütteln um ein kippen zu verhindern. Nach 5-6 Monaten abgießen, mehrfach filtrieren um Schwebstoffe und Verunreinigungen zu entfernen, die in Wunden zu Infektionen führen könnten.

\subsection{Anmerkungen}

Wird das Öl oral eingenommen, wird die Wirkung der Anti-Baby-Pille aufgehoben.\\ \\
Johanniskraut ist Phototoxisch, das heißt bei Hautkontakt mit dem Wirkstoff besteht ein stark erhöhtes Risiko für Sonnenbrand und Sonnenverbrennungen.\\
Der rote Wirkstoff heißt Hypericin.


\section{Melissenöl}

\subsection{Anwendung}

\textbf{Stressreduktion:} \\ \\
\textbf{Massage:} \\ \\

\cite{heilkraeuterlexikon} \url{https://heilkraeuter.de/lexikon/melisse.htm}

\subsection{Herstellung}

Frische (Zitronen-)Melisseblätter sammeln, wenn es zuvor 3 Tage lang Sonneneinstrahlung gab. Die Blätter einen Tag lang trocknen lassen, um den Wassergehalt zu reduzieren. Danach in eine Karaffe geben, diese mit Öl füllen und mehere Monate an einen dunklen, warmen Ort stellen und täglich schütteln. Die Wärme sollte 40 Grad in der Karaffe nicht überschreiten.

\subsection{Anmerkungen}





\section{Spitzwegerichöl}

\subsection{Anwendung}

\textbf{Insektenstiche:} \\ \\
\textbf{Kleine Wunden:} \\ \\


\cite{swrhandwerkskunst} \url{https://www.youtube.com/watch?v=OQeWS5Ktl0Y}

\subsection{Herstellung}

\cite{swrhandwerkskunst} \url{https://www.youtube.com/watch?v=OQeWS5Ktl0Y}

\subsection{Anmerkungen}

Wirkt Wundverschließend und desinfizierend.



\section{Chiliöl}

\subsection{Anwendung}

\textbf{Muskelentspannung:}

\subsection{Herstellung}

\url{https://heilkraeuter.de/rezept/chili-oel.htm}

\subsection{Anmerkungen}

Capsaicin löst sich in Öl weniger gut als in Alkohol




\section{Lavendelöl}

\subsection{Anwendung}

\subsection{Herstellung}

\subsection{Anmerkungen}





\section{Beinwellöl}

\subsection{Anwendung}

\subsection{Herstellung}

\subsection{Anmerkungen}
