\chapter{Öle}

\section{Vorwort}

% TODO Einleitung

\lipsum[1-5]
\newpage


\section{Johanniskrautöl}

Auch Rotöl oder Blut genannt. Ein sehr potentes Heilmittel.\\
Man sagt, Johanniskraut speichert die Kraft der Sonne. Nicht ohne Grund wurde die Pflanze Baldur, dem Gott des Lichtes, zugeordnet. \\
Wenn man die Blätter gegen das Licht hält, sind viele kleine Öldrüsen sind zu sehen. In diesen Drüsen ist ein roter Saft enthalten, der auch in den Blüten zu finden ist. Wenn du eine Blüte zwischen den Fingern zerreibst, tritt die färbende Flüssigkeit hervor. Sie speichert das Sonnenlicht, die Kraft der Sonne.\\
Das gespeicherte Sonnenlicht können wir mit Hilfe von Öl herauslösen und so in der sonnenarmen Zeit für uns nutzen. Mittlerweile ist Johanniskraut in vielen Medikamenten gegen Angstzustände und Depressionen enthalten. Es bringt Licht in unser Gemüt, welches wir vor allem in der dunklen Jahreszeit gut gebrauchen können.\\
Ich gebe das Öl gern ins Badewasser, etwa einen Esslöffel für ein Vollbad, oder reibe mir an besonders finsteren Tagen das Gesicht damit ein, denn Johanniskraut macht unsere Haut für Licht durchlässiger.\\
Das rote Öl besitzt auch für Wunden, insbesondere Verbrennungen, entzündeter Haut und der Erneuerung von Gewebe eine große Heilkraft.

% TODO Quellen
\url{https://www.kostbarenatur.net/rezepte/johanniskrautoel-einfach-selbst-herstellen/}

\url{https://www.chemie.de/lexikon/Echtes_Johanniskraut.html}

\index{Winterdepression} \index{Lichtmangel} \index{Depressive Stimmung} \index{Schlaflosigkeit} \index{Angstzustände} \index{Blutergüsse}
\index{Entzündungen} \index{Prellungen} \index{Wunden} \index{Juckreiz} \index{Sonnenbrand} \index{Schmerzen} \index{Schwellungen} \index{Wundheilung} \index{Wundversorgung} \index{Massage} \index{Muskelentspannung} \index{Winterdepression} \index{Antidepressivum} \index{Narbenentfernung} \index{Narben} \index{Verspannungen}

\subsection{Anwendung}

\textbf{Wundheilung:} Bei oberflächlichen Wunden wie z.B. Schürfwunden die Wunde reinigen, eine ES-Kompresse mit dem Öl tränken, auf die Wunde aufbringen und fixieren. Nach ca 5 Stunden den Verband wechseln. Fortfahren nach den üblichen Regeln der Wundversorgung. Bei infizierten Wunden mit Povidiod behandeln oder einen Arzt aufsuchen.\\ 

\textbf{Muskelentspannung:} Das Öl auf die betroffenen Körperregionen aufbringen und einmassieren, am besten mit einer Massage. \\ 

\textbf{Winterdepression:} Täglich etwas Öl auf der Haut an beliebigen Körperstellen wie zB. Arme, Schultern, Rücken verteilen und einmassieren. Das Johanniskraut bewirkt die Produktion von Vitamin D in der Haut. Wichtig, die entsprechenden Körperteile nicht direkter Sonne aussetzen, da eine stark Erhöhte Verbrennungsgefahr besteht! \\ 

\textbf{Antidepressivum:} Täglich 10-20 Tropfen zusammen mit Tee oder einem anderen Getränk einnehmen. Auch als Badezusatz geeignet. \\ 

\textbf{Narbenentfernung:} Morgens und Abends die Narbe im Wechsel mit einer Ringelblumensalbe und Rotöl einreiben, dies über einen längeren Zeitraum. Rotöl tendiert dazu, die Haut auszutrocknen, daher die Ringelblumensalbe.\\

\textbf{Sonnenbrand:} Die verbrannte stelle großzügig mit dem Öl einreiben, dabei das Öl gut verteilen. Wichtig, hiernach nicht mehr in die SOnne gehen, da Phototoxisch! Für eine noch bessere Wirkung Rotöl 1:1 mit Schafgarbenöl mischen.

\subsection{Herstellung}

Frisches Johanniskraut ist am 21. Juni zu sammeln. Die Pflanze enthält viele kleine rote Punktre, das ist der Wirkstoff. Man nehme promär die Blüten, entferne also das Kraut untenrum grob, trockne diese ein paar Stunden leicht an und gebe sie in eine Karaffe mit Korken. Die Karaffe wird nun mit Öl, idealerweise Olivenöl aufgefüllt, der Korken in Frischhaltefolie eingewickelt, und das ganze an einem warmen, sonnigen Ort deponiert. Jeden Tag schütteln um ein kippen zu verhindern. Nach 5-6 Monaten abgießen, mehrfach filtrieren um Schwebstoffe und Verunreinigungen zu entfernen, die in Wunden zu Infektionen führen könnten.

\subsection{Anmerkungen}

Wird das Öl oral eingenommen, wird die Wirkung der Anti-Baby-Pille aufgehoben.\\
Ebenso darf Johanniskraut nicht bei gleichzeitiger Einnahme von Herzmedikamenten angewendet werden.\\ 
Johanniskraut ist Phototoxisch, das heißt bei Hautkontakt mit dem Wirkstoff besteht ein stark erhöhtes Risiko für Sonnenbrand und Sonnenverbrennungen.\\
Der rote Wirkstoff heißt Hypericin. EIn weiterer Wirkstoff ist Hyperforin. \\
Nicht bei Hepatitis C. Es hilft nicht (auch wenn manche Quellen anderes sagen), es schadet im Gegenteil noch mehr.\\
Hypericin und Hyperforin sind geringe bis mittelstarke Wiederaufnahmehemmer von Serotonin, Noradrenalin und Dopamin und damit milde Antidepressiva. Eine Wirkung als Antidepressivum kann erst nach mehreren Wochen der Einnahme festgestellt werden.\\
Johanniskraut innerlich aktiviert das Enzym Cytochrom P450 in der Leber. Dadurch werden bestimmte Hormone und Medikamente schneller abgebaut. Auswirkungen davon sind unter Anderem die reduzierte Wirkung von hormonellen Verhültungsmitteln, aber auch die zumindest reduzierte Wirkung bestimmter AIDS-Medikamente, Immunsuppressiva sowie anderen Medikamenten wie z.B. Diazepam, Ciclosporin und Amiodaron.



\newpage




\section{Melissenöl}

Zitronenmelisse wirkt beruhigend und krampflösend. Bereits seit Jahrhunderten wird die Melisse als vielseitige Heilpflanze genutzt. Sie hat eine entspannende Wirkung auf das Nervensystem, wirkt beruhigend und lindert Ängste und Schlafstörungen. Bei Erkältungsbeschwerden wie Kopfschmerzen, Fieber oder Halsschmerzen kann sie ebenfalls heilend wirken. Äußerlich kann die Melisse in Form eines Ölauszuges auf der Haut angewendet werden. Dieser wirkt beruhigend, kühlend, schmerzlindernd und kann durch die antibakteriellen Wirkstoffen Beschwerden wie Insektenstiche, Juckreiz, gerötete und gereizte Haut sowie Entzündungen lindern. Auch als Massageöl ist es wegen seiner entspannenden, beruhigenden und krampflösenden Wirkung gut geeignet.

% TODO Quellen
\cite{heilkraeuterlexikon} 

\url{https://heilkraeuter.de/lexikon/melisse.htm} 

\url{https://www.kostbarenatur.net/anwendung-und-inhaltsstoffe/zitronenmelisse/}

\index{Massage} \index{Muskelentspannung}

\subsection{Anwendung}

Für alle Anwendungen sollte das Öl nur äußerlich angewendet werden. Die betreffende Körperregion ist damit einzureiben oder zu massieren.

\subsection{Herstellung}

Frische (Zitronen-)Melisseblätter sammeln, wenn es zuvor 3 Tage lang Sonneneinstrahlung gab. Die Blätter einen Tag lang trocknen lassen, um den Wassergehalt zu reduzieren. Danach in eine Karaffe geben, diese mit Öl füllen und mehere Monate an einen dunklen, warmen Ort stellen und täglich schütteln. Die Wärme sollte 40 Grad in der Karaffe nicht überschreiten.

%\subsection{Anmerkungen}




\newpage



\section{Schafgarbenöl}

Die Schafgarbe hat durch ihre vielen wertvollen Inhaltsstoffe eine wundheilende, blutstillende und entkrampfende Wirkung. Diese Eigenschaften sind besonders bei Menstruationsproblemen sehr hilfreich, was die Schafgarbe zu einer der wichtigsten Frauenheilkräutern macht.


% TODO Quellen
\url{https://www.kostbarenatur.net/rezepte/schafgarbe-mazerat-herstellen-gegen-menstruationsschmerzen-sonenbrand/}


\index{Krämpfe} \index{Kopfschmerzen} \index{Sonnenbrand} \index{Cellulite} \index{Mentstruationsbeschwerden} \index{Migräne}

\subsection{Anwendung}

\textbf{Krämpfe:} Die entsprechende Stelle wird mit dem Öl massiert.\\ 

\textbf{Kopfschmerzen \& Migräne:} Auch bei Kopfschmerzen und Migräne hilft eine Massage mit diesem Öl. Dazu werden Stirn, Schläfen und Nacken damit eingerieben. \\

\textbf{Menstruationsbeschwerden:} Die Schafgarbe zählt zu den wichtigsten Frauenheilkräutern, die ätherischen Öle wirken krampflösend, schmerzlindernd und duchblutungsfördernd. Bei zu schwacher oder unregelmäßiger Menstruation und Menstruationsschmerzen kann sie helfen, hierfür wird der Unterleib mit dem Öl massiert.\\

\textbf{Sonnenbrand:} In Kombination mit dem wertvollen Johanniskrautöl erhältst du ein effektives Mittel gegen Sonnenbrand. Die wundheilende Wirkung der Schafgarbe wird um die heilende Wirkung des Johanniskrautes bei Verbrennungen und Entzündungen ergänzt und hilft deiner Haut auf natürliche Weise. Verrühre dazu die beiden Öle zu gleichen Teilen und trage sie auf die geröteten Hautpartien auf. Durch diese Prozedur wird deine Haut aber auch lichtempfindlicher und sollte nicht mehr dem Sonnenlicht ausgesetzt werden.\\ 

\textbf{Cellulite:} Durch die Flavonoide dieser beiden Kräuter hilft die Mischung auch gegen Cellulite und fördert die Durchblutung. Massiere dazu die betroffenen Stellen regelmäßig mit diesem Öl oder behandle sie mit einer, aus diesem Öl selbst hergestellten, Schafgarbensalbe oder einem Peeling. \\


\subsection{Herstellung}

Die Schafgarbe sollte frisch an einem sonnigen Tag gesammelt werden, nachdem an den drei vorhergegangen Tage ebenfalls sonnig waren. Beim Sammeln sollten die Büschel kurz ausgeschüttelt werden, um Käfer und sonstige Insekten auszuschütteln. Nicht abspülen! Danach werden vorzüglich die Blüten in ein Gefäß gegeben und mit Öl übergossen. Danach muss das Gefäß verschlossen und an einem warmen, dunklen Ort deponiert werden. Täglich schütteln und ca. 3-5 Monate ausziehen lassen. Danach abgießen, das Kraut auspressen, filtrieren und in einem ausgekochten, dunklen Gefäß lagern. 

%\subsection{Anmerkungen}



\newpage



\section{Gänseblümchenöl}

Jeder kennt die kleine Blume, doch kaum einer weiß um die vielen möglichen Anwendungen des Gänseblümchens. Mit Vitamin A und C, vielen Gerb-, Bitter- und Schleimstoffen und den Mineralien Eisen, Kalium, Kalzium und Magnesium bietet es eine Reihe von wichtigen Inhaltsstoffen, die innerlich und äußerlich angewandt bei kleinen Verletzungen genauso wie bei Ausschlägen und Hautunreinheiten helfen können. Als Wund- und Heilmittel steht es Arnika und Ringelblume in nichts nach. Nicht umsonst wird das Gänseblümchen auch “Arnika des Nordens” genannt.

Ein Ölauszug mit Gänseblümchen wirkt vor allem äußerlich aufgetragen Wunder. So kann bei kleinen Wunden oder Schnitten ein Auftupfen auf die Haut eine schnelle Schmerzlinderung und ein Abklingen der Wunde bewirken. Dazu einfach mehrmals täglich die betroffene Stelle mit einem in Öl getränkten Tuch abtupfen.

Die im Gänseblümchen enthaltenen Gerb- und Bitterstoffe machen die Haut geschmeidiger und können so zum Beispiel helfen, Schwangerschaftsstreifen vorzubeugen. Eine regelmäßige Ölmassage der für Dehnungsstreifen anfälligen Stellen hilft den unschönen Hautrissen entgegenzuwirken.

Es kann auch als Öl zur Hautreinigung genutzt werden. Dabei das Öl vorsichtig auf der trockenen Haut verreiben und mit einem feuchten Tuch abnehmen. Zur Hautpflege wird das Öl auf die nasse Haut aufgetragen und nur vorsichtig abgetupft. Die enthaltenen Gerb- und Bitterstoffe helfen mit, die Anfälligkeit für Bakterien und Pilze zu mindern und Unreinheiten und Trockenheit zu bekämpfen. So kann ein Öl aus Gänseblümchen selbst bei Akne, Neurodermitis oder Herpes unterstützend wirken.

% TODO Quellen
\url{https://www.kostbarenatur.net/rezepte/gaensebluemchen-mazerat-herstellen-wirkung-entzuendungshemmend/}

% TODO Index

\subsection{Anwendung}
% TODO
\subsection{Herstellung}
% TODO

%\subsection{Anmerkungen}



\newpage




\section{Spitzwegerichöl}

% TODO Einleiung

% TODO Quellen
\cite{swrhandwerkskunst} 

\url{https://www.youtube.com/watch?v=OQeWS5Ktl0Y}

\index{Insektenstiche} \index{kleine Wunden}

\subsection{Anwendung}
% TODO
\textbf{Insektenstiche:} \\ 

\textbf{Kleine Wunden:} \\

\subsection{Herstellung}
% TODO
\subsection{Anmerkungen}

Wirkt Wundverschließend und desinfizierend.




\newpage



\section{Chiliöl}

% TODO Einleitung

% TODO Quellen
\url{https://heilkraeuter.de/rezept/chili-oel.htm}

\url{https://www.chemie.de/lexikon/Capsaicin.html}

\index{Muskelentspannung} \index{Gelenkschmerzen} \index{Ischias} \index{Hexenschuss} \index{Zerrungen} \index{Verspannungen} \index{Nackenschmerzen} \index{Rückenschmerzen} \index{Handgelenksschmerzen} \index{Muskelkater} \index{Muskelschmerzen} \index{Kreislaufanregung} \index{Magenschwäche} \index{Rheuma} \index{Blähungen} \index{Verdauungsschwäche} \index{Appetitlosigkeit} \index{Appetitlosigkeit}

\subsection{Anwendung}
% TODO
\textbf{Muskelentspannung:}

\subsection{Herstellung}
% TODO
\subsection{Anmerkungen}

Capsaicin löst sich in Öl weniger gut als in Alkohol.



\newpage



\section{Lavendelöl}

% TODO Einleitung, Quellen, Index

\subsection{Anwendung}
% TODO
\subsection{Herstellung}
% TODO

%\subsection{Anmerkungen}



\newpage




\section{Beinwellöl}

% TODO Einleitung, Quellen, Index

\subsection{Anwendung}
% TODO
\subsection{Herstellung}
% TODO

%\subsection{Anmerkungen}



\newpage




\section{Krampföl}

% TODO Einleitung, Quellen, Index

\index{Muskelkrämpfe} \index{Menstruationsbeschwerden}

\subsection{Anwendung}

Bei starken Muskelkrämpfen, wie beispielsweise bei Tagekrämpfen, ist dieses Öl hervorragend geeignet. Der Unterleib wird dabei mit dem Öl vorsichtig massiert, um die Krämpfe zu lösen.

\subsection{Herstellung}

Die Kräuter frisch sammeln und leicht antrocknen oder hochwertige, getrocknete Kräuter verwenden. Alles in eine Karaffe geben, mit Olivenöl übergießen, verschließen und an einen dunklen, warmen Ort platzieren, täglich schütteln.

\begin{itemize}
	\item Fenchelsamen
	\item Johanniskraut
	\item Schafgarbe
	\item Melisse
	\item Frauenmantel
	\item Ringelblume
	\item Lindenblüten
\end{itemize}

%\subsection{Anmerkungen}
